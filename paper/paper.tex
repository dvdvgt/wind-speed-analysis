%%%%%%%% ICML 2023 EXAMPLE LATEX SUBMISSION FILE %%%%%%%%%%%%%%%%%

\documentclass{article}

% Recommended, but optional, packages for figures and better typesetting:
\usepackage{microtype}
\usepackage{graphicx}
\usepackage{subfigure}
\usepackage{booktabs} % for professional tables
\usepackage{comment}

\usepackage{tikz}
% Corporate Design of the University of Tübingen
% Primary Colors
\definecolor{TUred}{RGB}{165,30,55}
\definecolor{TUgold}{RGB}{180,160,105}
\definecolor{TUdark}{RGB}{50,65,75}
\definecolor{TUgray}{RGB}{175,179,183}

% Secondary Colors
\definecolor{TUdarkblue}{RGB}{65,90,140}
\definecolor{TUblue}{RGB}{0,105,170}
\definecolor{TUlightblue}{RGB}{80,170,200}
\definecolor{TUlightgreen}{RGB}{130,185,160}
\definecolor{TUgreen}{RGB}{125,165,75}
\definecolor{TUdarkgreen}{RGB}{50,110,30}
\definecolor{TUocre}{RGB}{200,80,60}
\definecolor{TUmagenta}{RGB}{175,110,150}
\definecolor{TUmauve}{RGB}{180,160,150}
\definecolor{TUbeige}{RGB}{215,180,105}
\definecolor{TUorange}{RGB}{210,150,0}
\definecolor{TUbrown}{RGB}{145,105,70}

% hyperref makes hyperlinks in the resulting PDF.
% If your build breaks (sometimes temporarily if a hyperlink spans a page)
% please comment out the following usepackage line and replace
% \usepackage{icml2023} with \usepackage[nohyperref]{icml2023} above.
\usepackage{hyperref}


% Attempt to make hyperref and algorithmic work together better:
\newcommand{\theHalgorithm}{\arabic{algorithm}}

\usepackage[accepted]{icml2023}

% For theorems and such
\usepackage{amsmath}
\usepackage{amssymb}
\usepackage{mathtools}
\usepackage{amsthm}

\setlength{\abovedisplayskip}{0pt}
\setlength{\belowdisplayskip}{0pt}
\setlength{\abovedisplayshortskip}{0pt}
\setlength{\belowdisplayshortskip}{0pt}


% if you use cleveref..
\usepackage[capitalize,noabbrev]{cleveref}

%%%%%%%%%%%%%%%%%%%%%%%%%%%%%%%%
% THEOREMS
%%%%%%%%%%%%%%%%%%%%%%%%%%%%%%%%
\theoremstyle{plain}
\newtheorem{theorem}{Theorem}[section]
\newtheorem{proposition}[theorem]{Proposition}
\newtheorem{lemma}[theorem]{Lemma}
\newtheorem{corollary}[theorem]{Corollary}
\theoremstyle{definition}
\newtheorem{definition}[theorem]{Definition}
\newtheorem{assumption}[theorem]{Assumption}
\theoremstyle{remark}
\newtheorem{remark}[theorem]{Remark}

\newcommand{\R}{\mathbb{R}}

% Todonotes is useful during development; simply uncomment the next line
%    and comment out the line below the next line to turn off comments
%\usepackage[disable,textsize=tiny]{todonotes}
\usepackage[textsize=tiny]{todonotes}


% The \icmltitle you define below is probably too long as a header.
% Therefore, a short form for the running title is supplied here:
\icmltitlerunning{Project Report Template for Data Literacy 2023/24}

\renewcommand{\vec}[1]{\begin{bmatrix}#1\end{bmatrix}}

\begin{document}

\twocolumn[
\icmltitle{An Analysis of Wind Energy Potential in the North Sea}

% It is OKAY to include author information, even for blind
% submissions: the style file will automatically remove it for you
% unless you've provided the [accepted] option to the icml2023
% package.

% List of affiliations: The first argument should be a (short)
% identifier you will use later to specify author affiliations
% Academic affiliations should list Department, University, City, Region, Country
% Industry affiliations should list Company, City, Region, Country

% You can specify symbols, otherwise they are numbered in order.
% Ideally, you should not use this facility. Affiliations will be numbered
% in order of appearance and this is the preferred way.
\icmlsetsymbol{equal}{*}

\begin{icmlauthorlist}
\icmlauthor{Mohammad Fadel Berakdar}{equal,first}
\icmlauthor{Gwendolyn Neitzel}{equal,second}
\icmlauthor{David Voigt}{equal,third}
\icmlauthor{Alireza Yahyanejad}{equal,fourth}
\end{icmlauthorlist}

% fill in your matrikelnummer, email address, degree, for each group member
\icmlaffiliation{first}{Matrikelnummer 6117917, mohammad-fadel.berakdar@student.uni-tuebingen.de, MSc Bioinformatics}
\icmlaffiliation{second}{Matrikelnummer 5425507, gwendolyn.neitzel@student.uni-tuebingen.de, MSc Mathematik}
\icmlaffiliation{third}{Matrikelnummer 5416770, david.voigt@student.uni-tuebingen.de, MSc Computer Science}
\icmlaffiliation{fourth}{Matrikelnummer 6645496, alireza.yahyanejad@student.uni-tuebingen.de, MSc Computer Science}

% You may provide any keywords that you
% find helpful for describing your paper; these are used to populate
% the "keywords" metadata in the PDF but will not be shown in the document
\icmlkeywords{Machine Learning, ICML}

\vskip 0.3in
]

% this must go after the closing bracket ] following \twocolumn[ ...

% This command actually creates the footnote in the first column
% listing the affiliations and the copyright notice.
% The command takes one argument, which is text to display at the start of the footnote.
% The \icmlEqualContribution command is standard text for equal contribution.
% Remove it (just {}) if you do not need this facility.

%\printAffiliationsAndNotice{}  % leave blank if no need to mention equal contribution
\printAffiliationsAndNotice{\icmlEqualContribution} % otherwise use the standard text.

\begin{abstract}
%\textcolor{magenta}{
%Put your abstract here. Abstracts typically start with a sentence motivating why the subject is interesting. Then mention the data, methodology or methods you are working with, and describe results.}
In the face of climate change, it is widely agreed that the energy production has to rely on more sustainable and renewable forms of harnessing energy. Offshore wind turbine parks play a crucial role in increasing the share of green energy. This paper explores probabilistic methods of assessing the wind energy potential and potential trends of data collected on Helgoland by considering wind speeds as a Weibull distributed random variable. Further, for forecasting, the monthly expected wind power density is extrapolated using a Gaussian process regression model.
\end{abstract}

\section{Introduction}\label{sec:intro}
With the rising need for clean and sustainable energy due to climate change, offshore wind energy is playing an important role in the future energy mix of many European countries \citep{eu_comm}.
In the context of Germany, the North Sea is considered a prime location for offshore wind parks, with 41 farms already established and many more in planning \citep{wind-farms}.
For assessing the viability of offshore wind parks, an often used metric is the wind power density \cite{wang2021wind, miao2020117382, mohammadi2016assessing}.
In this paper, we follow \citet{mohammadi2016assessing} by considering the wind speed as a Weibull distributed random variable (Section \ref{sec:weibull}).
Further, we conduct a trend analysis (Section \ref{sec:weibull}) on the change of the parameters of the yearly wind speed distributions and look for evidence for the phenomenon of global terrestrial stilling, that is, a decrease in the global mean annual surface wind speed starting in the 1980s \cite{stilling}.
As proposed by \citet{mohammadi2016assessing}, we compute the wind power density probabilistically as the expectation of the monthly and yearly Weibull distributions (Section \ref{sec:power-density}). We then employ a Gaussian process regression model for the prediction of future wind power densities (Section \ref{sec:gp}). In Section \ref{sec:results}, we present the results of applying the aforementioned methods to real-world data.\footnote{The complete source of this paper can be found in our \href{https://github.com/FadelBerakdar/DataLiteracyProject}{repositoriy}.}

\iffalse
\section{Theoretical Background}

\subsection{Methods to Estimate the Parameters of the Weibull Distribution}

For estimating the two parameters of the Weibull distribution there are several methods available \citep{compestimators},  
of which we focus on three: The \emph{graphical method}, the \emph{maximum likelihood method} and the \emph{power density method}, also called \emph{energy pattern method}. \\
Let $(v_i)_{i \in \{1, \dots, n\}}$ be the family of observed wind speeds. \\
The graphical method (GM) exploits the fact that, after some transformations, 
equation \ref{eq:cdf} can be considered a linear equation: 
\begin{align}
    \underbrace{\ln ( - \ln (1- F(v))}_{:= y}= \beta \underbrace{\ln(v)}_{:=x}- \beta \ln(\lambda)
\end{align}
Then for all observed values $v_i$, the empirical CDF can be computed, 
such that we obtain the values $x_i$ and $y_i$, on which linear regression can be performed.  
Historically, the graphical method was convenient, because the regression line could be
determined by hand, avoiding heavy computation \citep{graphical}. \\

The maximum likelihood method (MLM) is a general method to estimate the parameters of any distribution, provided there are i.i.d. samples available.
This theoretically poses a hindrance to application in our case, as the wind speed measurements are clearly not i.i.d. . However, the method is still recommended and yields good results \citep{compestimators, graphical}.\\
Application of the maximum likelihood method to the Weibull distribution yields the equations
\begin{align}
    \frac{1}{\beta}=
    \left( \frac{\sum_{i=1}^n v_i^\beta\ln(v_i)}{\sum_{i=1}^n v_i^\beta}
    - \frac{\sum_{i=1}^n \ln(v_i)}{n}
    \right)
    \label{eq: beta}
\end{align}
and 
\begin{align}
    \lambda = \left( \frac{\sum_{i=1}^n v_i^\beta}{n}  \right)^{\frac{1}{\beta}}.
    \label{eq: lambda}
\end{align}
Then $\beta$ is chosen as the numerical solution to \ref{eq: beta}
(as there could be several solutions to \ref{eq: beta}, experience shows that the ones near $\beta=2$ are preferable \citep{newestimator})
and $\lambda$ is subsequently computed from \ref{eq: lambda}. \\

For the power density method (PDM) the mean and the third moment are used to derive 
the energy pattern factor
\begin{align}
     E_{pf}:= \frac{E[X^3]}{E[X]^3}=\frac{ \Gamma \left( 1 + \frac{3}{\beta} \right)}{ \Gamma \left( 1 + \frac{1}{\beta} \right)^3}.
\end{align}
The mean and third moment can be estimated 
by $\bar v= \sum_{i=1}^n v_i$ and $\bar{v^3}= \sum_{i=1}^n  v_i^3$ respectively. 
Then $\beta$ could be found numerically; However, \citep{newestimator} 
proposes 
\begin{align}
    \beta= 1 + \frac{3.69}{E_{pf}^2}
\end{align}
as an approximate solution, and thereby introduced the power density method. 
Subsequently, $\lambda$ is the solution to $\bar v \approx E[X]=\Gamma \left( 1 + \frac{1}{\beta} \right)$.
The advantage of this method lies in its simplicity, as it does not require an iterative procedure. 

\subsubsection{Goodness-of-Fit Criteria}
To evaluate the goodness-of-fit of the Weibull distribution to the wind speed frequencies, 
we first encounter the problem that we only have a finite number of samples of wind speeds, 
whereas the Weibull PDF is a continuous function.
This issue can be resolved by binning the samples and thus obtaining a categorical distribution. 
Then, for each bin, the empirical probability $p_i$ can be compared to the probability $\hat{p_i}$ obtained by the model. 
To obtain an appropriate number of bins, we apply the square root rule.\\
As a criterion for goodness-of-fit, the literature
\citep{review, compestimators,newestimator} often uses the root-mean-square error (RMSE),
\begin{align}
    \mathrm{RMSE}=\sqrt{ \frac{1}{k}\sum_{i=1}^k(p_i-\hat p_i)^2}
\end{align}
treating the problem as that of an ordinary regression.
However, as the issue at hand is the fitting of a probability distribution, the Kullback-Leibler Divergence,
comparing the ground-truth distribution $P$ to the estimated distribution $\hat P$,
    \begin{align}
        \mathrm{D_{KL}}(P, \hat P)=\sum_{i=1}^k \hat p_i \cdot \log\left(\frac{p_i}{\hat p_i}\right)
    \end{align}
is especially suitable.\\
\textcolor{magenta}{need citations for this?}

\fi


\section{Data}\label{sec:data}
%\textcolor{magenta}{In this section, describe \emph{what you did}. Roughly speaking, explain what data you worked with, how or from where it was collected, it's structure and size.Explain your analysis, and any specific choices you made in it. Depending on the nature of your project, you may focus more or less on certain aspects. If you collected data yourself, explain the collection process in detail. If you downloaded data from the net, show an exploratory analysis that builds intuition for the data, and shows that you know the data well. If you are doing a custom analysis, explain how it works and why it is the right choice. If you are using a standard tool, it may still help to briefly outline it. Cite relevant works.}You can use the \verb|\citep| and \verb|\citet| commands for this purpose \citep{mackay2003information}.

The wind speed data used in this paper was collected by \href{https://www.dwd.de}{DWD} on \href{https://www.openstreetmap.org/?mlat=54.1750&mlon=7.8920#map=15/54.1750/7.8920}{Helgoland}, spanning the years from 1996 to 2022. The wind speed is recorded at a temporal resolution of ten-minutes\footnote{Averaged over the last ten-minute interval.}, with the anemometer being $4.38 \mathrm{m}$ above ground. Averaged over the whole measurement period, the data has a yearly mean completeness of about $91\%$. Notably, the year 2019 is missing $88\%$ of its measurements. We consider this, and a noticeable drop in the yearly wind speed mean and variance of the following years, as evidence that the measurement process was changed, and therefore discard them. Notably, the years 1996 to 1998 have a completeness of about $3\%, 85\%$ and $84\%$, respectively. For preventing a distorting effect on the Weibull distribution estimations, we also did not consider these years for our analysis, thereby improving the yearly mean completeness to $99\%$. For handling outliers, we dropped measurements exhibiting a z-score $> 8$.

\section{Methods}\label{sec:methods}

\subsection{Fitting of the Weibull Distribution and Trend Analysis}\label{sec:weibull}
Based on the observed wind data, we model the wind speed probability distribution both on a monthly and yearly basis.
Whilst there are several other models, like the three-parameter Weibull distribution or the normal distribution for fitting wind speed frequencies, the most prominent one is the two-parameter Weibull distribution
\cite{review, statanalysis, mohammadi2016assessing}.
For the scale parameter $\lambda >0$ and shape parameter $\beta >0$, the probability density function (PDF) and cumulative density function (CDF) are given by
\begin{align}
    \label{eq:pdf}
    p(v) &=
    \begin{cases}
        \frac{\beta}{\lambda} \left( \frac{v}{\beta}\right)^{\beta-1} 
        \exp{ \left( -\left(\frac{v}{\lambda}\right)^\beta \right)} & \text{  for } v>0 \\
        0 & \text{ else}
    \end{cases} \\
    \label{eq:cdf}
    P(v)&=
    \begin{cases}
        1- \exp{\left(-\left(\frac{v}{\lambda}\right)^\beta \right)} & \text{  for } v>0 \\
        0 & \text{ else}
    \end{cases}.
\end{align}
Via the gamma function, $\Gamma(z)= \int_{0}^{\infty}t^{z-1} \mathrm{e}^{-t}dt$, 
$k$-th raw moment and variance of a random variable $V$ with Weibull distributions can be expressed as follows: 
\begin{align}  
    \mathrm{E}[V^k] &= \lambda^k \Gamma \left( 1 + \frac{k}{\beta} \right) \label{eq:n-raw-moment} \\
    \mathrm{Var}[V] &= \lambda^2 \left[  \Gamma \left( 1 + \frac{2}{\beta}\right) \label{eq:var}
    - \Gamma^2 \left( 1 + \frac{1}{\beta}\right)\right].
\end{align}

We determine the parameters $\lambda$ and $\beta$ for every month and year using 
%\textcolor{magenta}{suitable estimator and assess the goodness of fit}. \\
the maximum likelihood method applied to \eqref{eq:pdf} \cite{mohammadi2016assessing}.
Due to the seasonal variability of wind speed, the long-term trend analysis is conducted on the yearly 
(rather than monthly) estimated wind speed PDFs. For this, we employ a least-squares linear regression and assess the statistical significance through a two-sided permutation test at the $5\%$ level. 
Since the two Weibull parameters fully characterize the distribution, we are then able to use this analysis to infer trends in other quantities of interest, like mean and standard deviation.

\subsection{Wind Power Density}\label{sec:power-density}

The wind power density is often used as an approximation of the physical upper limit of how much energy a wind turbine is able to harness \cite{wang2021wind, mohammadi2016assessing, miao2020117382}. The power density is given by
\begin{align}\label{eq:power}
    \mathrm{P} &= \frac{1}{2} \cdot \rho \cdot \mathrm{v}^3 \\
    [\mathrm{P}] &= [\rho] \cdot [\mathrm{v}^3] = \frac{\mathrm{kg}}{\mathrm{m}^3} \cdot \frac{\mathrm{m}^3}{\mathrm{s}^3} = \frac{\mathrm{W}}{\mathrm{m}^2}
\end{align}
where $\rho$ is the air density, and $\mathrm{v}$ is the wind speed. Notice that the wind power density is independent of the area covered by the wind turbine's rotors. Using equation \eqref{eq:power} the expected power density can be found by calculating the expected value under the given distribution $p(\mathrm{v})$ as follows:
\begin{align}\label{eq:power-expectation}
    E[P] &= \int_{\mathrm{v}_\mathrm{min}}^{\mathrm{v}_\mathrm{max}} \frac{1}{2} \cdot \rho \cdot \mathrm{v}^3 \cdot p(\mathrm{v}) \, \mathrm{dv} \\
    &= \frac{1}{2} \cdot \rho \cdot \int_{\mathrm{v}_\mathrm{min}}^{\mathrm{v}_\mathrm{max}} \mathrm{v}^3 \cdot p(\mathrm{v}) \, \mathrm{dv}.
\end{align}
For a range of $v_\mathrm{min} = 0 \, \frac{\mathrm{m}}{\mathrm{s}}$ and $v_\mathrm{max} = \infty \, \frac{\mathrm{m}}{\mathrm{s}}$, the solution to this integral is given by the third raw-moment as shown in \eqref{eq:n-raw-moment}. However, in using these limits, we would assume that wind turbines are able to convert energy during arbitrarily high or low wind speeds. Thus,
a more sensible assumption is to use the so-called cut-in $\mathrm{v}_\mathrm{in}$ and cut-out wind speed $\mathrm{v}_\mathrm{out}$ of wind turbines. For a generic wind turbine, a common assumption is $\mathrm{v}_\mathrm{in} \in [2, 4] \, \frac{\mathrm{m}}{\mathrm{s}}$ and $\mathrm{v}_\mathrm{out} \in [20, 23] \, \frac{\mathrm{m}}{\mathrm{s}}$ \cite{dupont2017, wang2021wind}.

\subsection{Gaussian Process Regression}\label{sec:gp}
For forecasting, we employ a Gaussian process regression (GPR) model to perform a time-series regression on the monthly estimated power density and extrapolate future values. Thereby, we assume that the wind speed data $Y \in \mathbb{R}^N$ is from a latent function $f: \mathbb{R}^N \rightarrow \mathbb{R}^N$ with some added measurement noise: $y = f(x) + \epsilon$. The GPR framework consists of a Gaussian process as prior, multivariate Gaussian likelihood and Gaussian process posterior conditioned on the data 
\cite{rasmussen-williams-gp}:
\begin{align}
    \label{eq:gp_prior}
    p(f) &= \mathcal{GP}(f; 0, k), \\
    \label{eq:gp_likelihood}
    p(y \mid f) &= \mathcal{N}(Y; f, \sigma^2 I), \\
    \label{eq:gp_posterior}
    p(f \mid Y) &= \mathcal{GP}(f; m', k') \propto p(Y \mid f) \cdot p(f).
\end{align}
The resulting posterior (and prior) distribution collapses into a multivariate normal distribution when evaluate at some new locations $X' \in \mathbb{R}^M$.

\section{Results}\label{sec:results}
%\textcolor{magenta}{In this section, outline your results. At this point, you are just stating the outcome of your analysis. You can highlight important aspects (``we observe a significantly higher value of $x$ over $y$''), but leave interpretation and opinion to the next section. This section absolutely \emph{has} to include at least two figures.}\\

\iffalse
\begin{figure*}
    \subfigure[The linear regression of the yearly Weibull parameters shows a noticeable decrease in 
 \textcolor{TUred}{$\lambda$}, but less so in \textcolor{TUdarkblue}{$\beta$}.]
        {\includegraphics[width=0.475\textwidth]{fig/linreg_parameters.pdf}
        \label{fig:linreg}
    }
    \hfill
  \subfigure[The estimated Weibull PDF for each year and the fit to the empirical PDF in 2018]{
    \includegraphics[width=0.475\textwidth]{fig/yearly_weibull.pdf}
    \label{fig:yearly_weibull}
    }
  \caption{The yearly Weibull parameters in \ref{fig:linreg} correspond to the PDFs in \ref{fig:yearly_weibull}.}
\end{figure*}
\fi

\iffalse

\subsection{Weibull Parameter Estimation}

The goals of this subsection are two-fold: Firstly to assess the overall fit of the Weibull distribution to the observed wind speed frequencies, 
and secondly, which of the estimators presented in 
\textcolor{magenta}{\ref{subsubsec: estimators}}
is most suitable. 
Since the models for the wind speed distribution are set to be used for 
monthly and yearly time periods, the goodness-of-fit is also evaluated at monthly and yearly basis.
Then the mean of the respective RMSE values and KL-Divergence values are used as the metric of comparison. 
The results are depicted in Figure \ref{tab:metrics}.


\begin{table}
    \begin{center}
        \begin{tabular}{l || c|c|c}
           Criterion & GM & MLM & PDM \\
            \hline
             av. yearly RMSE & 0.0020 & 0.0020 & 0.0020 \\
             av. yearly KL-Divergence & 0.0499 & 0.0455 & 0.0456 \\
             av. monthly RMSE & 0.0057 & 0.0050 & 0.0053 \\
             av. yearly KL-Divergence & 0.0544 & 0.0436  & 0.0462
        \end{tabular}
    \end{center}
    \caption{The average goodness-of-fit is best for the maximum likelihood method (MLM), closely followed by the probability density method (PDM) and 
    worst for the graphical method (GM). }\label{tab:metrics}
\end{table}

To put the values of the RMSE into perspective: As the number of bins are chosen according to the square root rule, the expected probability 
of a bin for the yearly time frame is approximately $\frac{1}{[\sqrt{6 \cdot 24 \cdot 365 }]} \approx 0.44 \%$ and for the monthly time frame it is 
 $\frac{1}{[\sqrt{6 \cdot 24 \cdot 30 }]} \approx 1.52 \%$ .

We can conclude that the maximum likelihood method provides the best results, 
while the power density method performs only slightly worse. 
Hence, we chose the maximum likelihood estimator for use in further analyses. 

\fi


\subsection{Weibull Parameter Estimation and Trend Analysis}
\label{sec:trend}

For the available years (1999-2018) we estimated the parameters of the Weibull distribution on a monthly and yearly basis. 
\begin{figure}
    \centering 
    \includegraphics{fig/monthly_weibull.pdf}
    \caption{The monthly estimated Weibull distributions in the year 2010. The parameter $\lambda$ decreases in summer, whilst $\beta$ decreases, resulting in a shift to the left and lower standard deviation.}
    \label{fig:monthly_weibull}
\end{figure}
Figure \ref{fig:monthly_weibull} depicts the resulting PDFs for the months of 2010 as well 
as the empirical histogram in December.
When applying the square root rule to the number of bins,
 this fit has an expected relative error per bin of $18 \%$. 
 Averaged over the monthly and yearly distributions,
 the expected relative error per bin is  $22 \%$ and $23 \%$, respectively.

\begin{figure}
    \centering
    \includegraphics{fig/linreg_parameters.pdf}
    \caption{The linear regression of the yearly Weibull parameters shows a noticeable decrease in 
 \textcolor{TUred}{$\lambda$}, but less so in \textcolor{TUdarkblue}{$\beta$}.}
    \label{fig:lin_reg}
\end{figure}

The least-squares linear regression of the yearly parameters can be seen in Figure \ref{fig:lin_reg}.
It shows a decrease of the parameter $\lambda$ of 
$-0.059\frac{m}{s} \frac{1}{\mathrm{yr}}$ with an RMSE of $0.42\frac{m}{s}$. 
For $\beta$, the slope is $-0.002 \frac{1}{\mathrm{yr}}$, with an RMSE of $0.07  \frac{m}{s}$.
At the $5\%$ level of the two-sided permutation test, the null hypothesis that there is no linear trend in 
the development of the parameter $\lambda$ can be rejected (p-value$=0.002$), so the observation is statistically significant. 
For the parameter $\beta$, however, the p-value is $0.371$, so we cannot reject the null hypothesis.
Assuming the trend observed in $\lambda$ and the empiric mean $\bar \beta=2.28$ for $\beta$, we derive the following trends in other quantities:
As the mean  and standard deviation are proportional to $ \lambda$, we can infer a linear trend of $-0.052 \frac{m}{s}  \frac{1}{\mathrm{yr}}$
and  $-0.024 \frac{m}{s} \frac{1}{\mathrm{yr}}$, respectively. 
%Further, the probabilities of wind speed below $ 3 \frac{m}{s}$ and above $22 \frac{m}{s}$ are especially relevant (Section \ref{sec:power-density}):
%Over the entire $22$ year period, the estimated probabilities change (in a non-linear fashion) from $6.46 \%$ to $8.15 \%$ and from $0.17 \%$ to $0.03 \%$, respectively. 
Further, under these assumptions, the mean power density 
of the mean wind speed decreases (non-linearly) from 1999 to 2018  by $30\%$. 

\subsection{Expected Power Density}

By using \eqref{eq:power-expectation}, we numerically computed the expected power density under each of the estimated monthly Weibull distributions with $v_\mathrm{min} = 3 \, \frac{\mathrm{m}}{\mathrm{s}}$ and $v_\mathrm{max} = 22 \, \frac{\mathrm{m}}{\mathrm{s}}$. The resulting values can be found as the ground truth in Figure \ref{fig:gp-samples} and \ref{fig:gp-pred}. 

\begin{figure}
    \centering
    \includegraphics{fig/gp_samples.pdf}
    \caption{The \textcolor{TUdarkblue}{ground truth} data plotted together with three samples $f_i \sim \mathcal{GP}(f; 0, k)$ drawn from the Gaussian Process prior distribution with the added \textcolor{TUred}{mean} $\overline{Y}$ of the data $Y$.}
    \label{fig:gp-samples}
\end{figure}


\subsection{Gaussian Process Regression}
\begin{figure*}[htb!]
    \centering
    \includegraphics[width=0.95\textwidth]{fig/gp_pred.pdf}
    \caption{The \textcolor{TUred}{mean} and \textcolor{TUred}{standard deviation} of the posterior distribution $\mathcal{N}(f(x); m', k')$ evaluated at $x$ is plotted together with the linearly interpolated \textcolor{TUdarkblue}{ground truth}. Starting at the \textcolor{TUdarkgreen}{data cut-off}, the model is extrapolating (predicting) the power density into the future.}
    \label{fig:gp-pred}
\end{figure*}


%For assessing the viability of wind turbine (parks), one might also be interested in the weekly or daily values  and, even more importantly, in a forecast. Hence, for interpolating the daily expected power density, we employ a Gaussian Process Regression model to perform a time-series regression. In addition to interpolating, we also use the same model for extrapolation and thereby making predictions about the future daily expected power density.
The first step for conducting a GP regression is to decide on a kernel $k$ for the prior \eqref{eq:gp_prior}. By analysing the plotted raw monthly expected power density, we identify a yearly periodic trend and a slight global downward trend. The former can be captured by the Exponential-Sine-Squared kernel (ES) \cite{MacKay1998IntroductionTG}. Supported by the empirical data, we choose a periodicity of $365 \, \text{days}$ and an amplitude of $450 \, \frac{\textrm{W}}{\text{m}^2}$. For the length scale, we settle for a rather small value ($< 2 \, $ days) in order to capture local patterns in the periodicity more accurately. For allowing the amplitude to slowly decay over long periods of time, we multiply the ES kernel by a very smooth (length scale $> 50 \, \text{years}$) Squared-Exponential (SE) kernel. For reproducing the global downwards trend observed in section \ref{sec:trend}, we also choose a very smooth Squared-Exponential kernel with an output scale of $100 \, \frac{\textrm{W}}{\text{m}^2}$. For accounting for the monthly midterm trend, we add a Rational-Quadratic kernel \cite{rasmussen-williams-gp} with a scale of $10 \, \frac{\textrm{W}}{\text{m}^2}$ and a length scale of one month. Finally, the additive kernel, consisting of a White-Noise (WN) and a Matern kernel (MA) \cite{abramowitz1968handbook}, is responsible for explaining measurement noise in the data and thus also prevents overfitting. Thereby, the kernel has the form
$$k = \underbrace{\text{ES} \cdot \text{SE}}_{\text{yearly trend}} + \underbrace{\text{RQ}}_\text{monthly trend} + \underbrace{\text{ES}}_\text{longterm trend} + \underbrace{\text{WN} + \text{MA}}_{\text{noise}}.$$
By drawing samples\footnote{with the mean of the data added.} $f_i \sim \mathcal{GP}(f; 0, k)$ from the prior distribution \eqref{eq:gp_prior}, we gauge how accurately our kernel choice fits to the given data. The results are shown in Figure \ref{fig:gp-samples}. We can observe that the amplitude and periodicity of some of the samples mirrors the data's in an accurate manner, supporting our choice of hyperparameters. After conditioning the Gaussian Process prior on the normalised data, we obtain the posterior distribution $\mathcal{GP}(f; m', k')$ \eqref{eq:gp_posterior}. The evaluated posterior mean and standard deviation can be found in Figure \ref{fig:gp-pred}. We can observe that the mean function is following the periodicity of the ground truth without overfitting extreme power densities. Furthermore, the uncertainty of the model (standard deviation) increases gradually the further the prediction is from the data. Notably, the model also captures the global downward trend. When comparing the prediction to the test data in the years 2017 and 2018, we can observe that the prediction fits the ground truth accurately and all data points are well within one standard deviation of the mean.

\section{Discussion \& Conclusion}\label{sec:conclusion}
%\textcolor{magenta}{Use this section to briefly summarize the entire text. Highlight limitations and problems, but also make clear statements where they are possible and supported by the analysis.}
In this paper, we analysed the wind energy potential of the island Helgoland in the North Sea. 
We have found a decrease in the mean annual wind speed of 
$-0.052 \frac{m}{s}  \frac{1}{\mathrm{yr}}$, 
consistent with the global phenomenon of terrestrial 
stilling \cite{stilling}. In this location, a reversal in the stilling, as  was observed globally \cite{stilling-reversal}, could not be found. If the trend prevails, diminishing wind energy production is to be expected at this particular location.
Further, we have shone light on the monthly variations in expected power density, a valuable insight for evaluating potential contribution to the overall energy supply.
Lastly, we explored Gaussian process regression as a tool for extrapolating the expected power density.
We have found that it is adequate for this purpose, 
being able to adapt to a variety of subtrends and cyclical behaviours due to the explicit and flexible choice of the prior kernel function. By comparing the test data with the prediction, we consider the fit to be accurate and the uncertainty quantification realistic.

However, there are some shortcomings. One of the inherent limitations of the GPR is that it yields functions mapping onto $\R$, while the power density is only limited to $\R_+$. 
%Further, we conducted an interpolation of daily values based on monthly data, which does not take the inter-day variability into account. For the purpose of extrapolating daily power densities, 
%it might be advisable in further work to rely on daily data to begin with, complicating the choice of the kernel.  
Another important limitation is that our data was recorded at a height of $4.38 \mathrm{m}$, whereas the real height of wind turbines can exceed $100 \mathrm{m}$. One commonly applied rule of thumb to mitigate this issue
is multiplying the wind speeds by the factor 
$\left(\frac{h}{4.38 \mathrm{m}} \right)^{0.14}$ or the power density by its cube, 
where $h$ is the  proposed hub height of a turbine \citep{statanalysis}. However, this is far from exact, as the winds at different heights could be subject to different circulations. 
Furthermore, we did not take the direction of the wind into account and made the simplification that wind turbines are always perfectly aligned. 
Also, for the \eqref{eq:power}, we assumed a constant air pressure rather than considering the monthly changing pressure.
Lastly, with the use of the maximum-likelihood method, we assumed the independence of measurements. While this is generally not given, we were still able to produce adequate estimations.

\section*{Contribution Statement}
David Voigt devised the wind power density and conducted the Gaussian process regression. 
Gwendolyn Neitzel was mainly concerned with the Weibull model fitting and the trend analysis.
Mohammad Fadel Berakdar focused on the data preprocessing. Alireza Yahyanejad worked on data loading.
All authors were involved in data exploration and jointly wrote the report. 
%\textcolor{magenta}{Explain here, in one sentence per person, what each group member contributed. For example, you could write: Max Mustermann collected and prepared data. Gabi Musterfrau and John Doe performed the data analysis. Jane Doe produced visualizations. All authors will jointly write the text of the report. Note that you, as a group, are collectively responsible for the report. Your contributions should be roughly equal in amount and difficulty.}

%\section*{Notes} 
%\textcolor{magenta}{
%Your entire report has a \textbf{hard page limit of 4 pages} excluding references. (I.e. any pages beyond page 4 must only contain references). Appendices are \emph{not} possible. But you can put additional material, like interactive visualizations or videos, on a GitHub repo (use \href{https://github.com/pnkraemer/tueplots}{links} in your PDF to refer to them). Each report has to contain \textbf{at least three plots or visualizations}, and \textbf{cite at least two references}. More details about how to prepare the report, including how to produce plots, cite correctly, and how to ideally structure your GitHub repo, will be discussed in the lecture, where a rubric for the evaluation will also be provided.
%}

\pagebreak

\bibliography{bibliography}
\bibliographystyle{icml2023}

\end{document}


% This document was modified from the file originally made available by
% Pat Langley and Andrea Danyluk for ICML-2K. This version was created
% by Iain Murray in 2018, and modified by Alexandre Bouchard in
% 2019 and 2021 and by Csaba Szepesvari, Gang Niu and Sivan Sabato in 2022.
% Modified again in 2023 by Sivan Sabato and Jonathan Scarlett.
% Previous contributors include Dan Roy, Lise Getoor and Tobias
% Scheffer, which was slightly modified from the 2010 version by
% Thorsten Joachims & Johannes Fuernkranz, slightly modified from the
% 2009 version by Kiri Wagstaff and Sam Roweis's 2008 version, which is
% slightly modified from Prasad Tadepalli's 2007 version which is a
% lightly changed version of the previous year's version by Andrew
% Moore, which was in turn edited from those of Kristian Kersting and
% Codrina Lauth. Alex Smola contributed to the algorithmic style files.
